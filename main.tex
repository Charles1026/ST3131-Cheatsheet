%%%%%%%%%%%%%%%%%%%%%%%%%%%%%%%%%%%%%%%%%%%%%%%%%%%%%%%%%%%%%%%%%%%%%%
% writeLaTeX Example: A quick guide to LaTeX
%
% Source: Dave Richeson (divisbyzero.com), Dickinson College
% 
% A one-size-fits-all LaTeX cheat sheet. Kept to two pages, so it 
% can be printed (double-sided) on one piece of paper
% 
% Feel free to distribute this example, but please keep the referral
% to divisbyzero.com
% 
%%%%%%%%%%%%%%%%%%%%%%%%%%%%%%%%%%%%%%%%%%%%%%%%%%%%%%%%%%%%%%%%%%%%%%
% How to use writeLaTeX: 
%
% You edit the source code here on the left, and the preview on the
% right shows you the result within a few seconds.
%
% Bookmark this page and share the URL with your co-authors. They can
% edit at the same time!
%
% You can upload figures, bibliographies, custom classes and
% styles using the files menu.
%
% If you're new to LaTeX, the wikibook is a great place to start:
% http://en.wikibooks.org/wiki/LaTeX
%
%%%%%%%%%%%%%%%%%%%%%%%%%%%%%%%%%%%%%%%%%%%%%%%%%%%%%%%%%%%%%%%%%%%%%%

\documentclass[a4paper, 10pt,landscape]{article}
\usepackage{amssymb,amsmath,amsthm,amsfonts}
\usepackage{multicol,multirow}
\usepackage{calc}
\usepackage{ifthen}
\usepackage[landscape]{geometry}
\usepackage[colorlinks=true,citecolor=blue,linkcolor=blue]{hyperref}
\usepackage{ragged2e}


\ifthenelse{\lengthtest { \paperwidth = 11in}}
    { \geometry{top=.5in,left=.5in,right=.5in,bottom=.5in} }
	{\ifthenelse{ \lengthtest{ \paperwidth = 297mm}}
		{\geometry{top=1cm,left=1cm,right=1cm,bottom=1cm} }
		{\geometry{top=1cm,left=1cm,right=1cm,bottom=1cm} }
	}
\pagestyle{empty}
\makeatletter
\renewcommand{\section}{\@startsection{section}{1}{0mm}%
                                {-1ex plus -.5ex minus -.2ex}%
                                {0.5ex plus .2ex}%x
                                {\normalfont\large\bfseries}}
\renewcommand{\subsection}{\@startsection{subsection}{2}{0mm}%
                                {-1explus -.5ex minus -.2ex}%
                                {0.5ex plus .2ex}%
                                {\normalfont\normalsize\bfseries}}
\renewcommand{\subsubsection}{\@startsection{subsubsection}{3}{0mm}%
                                {-1ex plus -.5ex minus -.2ex}%
                                {1ex plus .2ex}%
                                {\normalfont\small\bfseries}}
\makeatother
\setcounter{secnumdepth}{0}
\setlength{\parindent}{0pt}
\setlength{\parskip}{0pt plus 0.5ex}

\newcommand{\boldbeta}{\boldsymbol{\beta}}
\newcommand{\boldeps}{\boldsymbol{\epsilon}}

\newcommand{\X}{\mathbf{X}}
\newcommand{\y}{\mathbf{y}}
\newcommand{\E}{\mathbb{E}}
% -----------------------------------------------------------------------

\title{ST3131 Cheatsheet}

\begin{document}

\justifying
\noindent
\footnotesize

\begin{multicols*}{3}
\setlength{\premulticols}{1pt}
\setlength{\postmulticols}{1pt}
\setlength{\multicolsep}{1pt}
\setlength{\columnsep}{2pt}

\section{2.1 Simple Linear Regression}
$y = \beta_0 + \beta_1 x + \epsilon$ is the simple linear regression model where $\beta_0$, the intercept, and $\beta_1$, the slope, are constant and $\epsilon$ is the random noise, modeled as independent random variables with $\mu = 0$ and $Var = \sigma^2$. Hence, $x$ is the regressor and y is a random variable so $\mathbb{E}(y|x) = \beta_0 + \beta_1 x$ and $Var(y|x) = \sigma^2$. We see that the mean of $y$ is a linear function through $x$ and the variance is not dependent on $x$.

\section{2.2 Least Squares Estimate}
\subsection{2.2.1 Estimation of $\beta_0$ \& $\beta_1$}
$\hat{\beta_0} = \bar{y} - \hat{\beta_1} \bar{x}$ and $\hat{\beta_1} = \frac{S_{XY}}{S_{XX}}$ are the least squares estimators of the parameters, derived from differentiating and solving the least squares equation $S(\beta_0, \beta_1 ) = \sum ( y_i - \beta_0 - \beta_1 x_i )^2$ w.r.t $\beta_0$ and $\beta_1$. Here $S_{XY} = \sum y_i (x_i - \bar{x}) = \sum x_i (y_i - \bar{y})$ and $S_{XX} = \sum(x_i - \bar{x})^2$.

\subsection{2.2.2 Properties of $\beta_0$ \& $\beta_1$}
We see that $\hat{\beta_1} = \sum \frac{x_i - \bar{x}}{S_{XX}} y_i$ and $\hat{\beta_0}$ are linear combinations of of $y_i$ and can prove that they are unbiased estimators by taking their expectations. The variances of the estimators are $Var(\hat{\beta_1}) = \frac{\sigma^2}{S_{XX}}$ and $Var(\hat{\beta_0}) = \sigma^2 (\frac{1}{n} + \frac{\bar{x}^2}{S_{XX}})$. By the Gauss-Markov theorem, with i.i.d. $\epsilon$, $\mathbb{E}(\epsilon) = 0$ and $Var(\epsilon) = \sigma^2$ the least squares estimators are unbiased and have minimum variance when compared to all other estimators. Other properties are that $\sum e_i = 0$, $\sum y_i = \sum \hat{y_i}$ and $\sum x_i e_i = \sum y_i e_i = 0$.

\subsection{2.2.3 Variance Estimation $\sigma^2$}
The variance $\sigma^2$ is estimated as $SS_Res = \sum e_i^2$ as we do not have several observations of $y_i$ for at least one $x_i$. If we do, we can estimate the variance independent of the model's accuracy. We see that $SS_T = \sum (y_i - \bar{y})^2 = SS_{Res} + \hat{\beta_1} S_{XY}$, where $SS_R = \hat{\beta_1} S_{XY}$. $SS_{Res}$ has $n - 2$ degrees of freedom due to $\hat{\beta_0}$ and $\hat{\beta_1}$ being used to calculate $\hat{y_i}$. Hence, $\hat{\sigma}^2 = \frac{SS_{Res}}{n - 2} = MS_{Res}$ is an unbiased estimator $\sigma^2$.

\subsection{2.2.4 Alternate Form of Model}
We can also write the model as $y_i = \beta_0' + \beta_1(x_i - \bar{x}) + \epsilon_i$, where $beta_0' = beta_0 + beta_1 + \bar{x}$ which implies $beta_0' = \bar{y}$. This centers the model around $\bar{x}$ and makes the least squares estimators uncorrelated, good for finding confidence intervals.

\section{2.3 Hypothesis Testing on $\beta_0$ \& $\beta_1$}
\subsection{2.3.1 t Tests}
We test for $H_0: \beta_1 = \beta_{test_i}$ and $H_1: \beta_1 \ne \beta_{test_i}$ with $Z_0 = \frac{\hat{\beta_1} - \beta_{test_i}}{\sqrt{\sigma^2 / S_{XX}}}$ and $t_0 = \frac{\hat{\beta_1} - \beta_{test_i}}{\sqrt{MS_{Res} / S_{XX}}}$ if the $\sigma^2$ is unknown and reject $H_0$ if $|t_0| > t_{\alpha/2, n - 2}$. For $\beta_0$ we replace the denominator with $MS_{Res} (\frac{1}{n} + \frac{\bar{x}^2}{S_{XX}})$. The denominators are known as the standard errors($se$) of the estimators.

\subsection{2.3.2 t Testing Significance of Regression}
By testing $H_0: \beta_1 = 0$ and $H_1: \beta_1 \ne 0$, we can find if $x$ and $y$ have a linear relationship, says nothing about the presence or lack of other relationships. The formula replaces the $\beta_{test_i}$ of the t test with $0$.

\subsection{2.3.3 Analysis of Variance}
$SS_T = SS_R + SS_{Res}$ is $\sum (y_i - \bar{y})^2 = \sum (\hat{y_i} - \bar{y})^2 + \sum (y_i - \hat{y_i})^2$. $SS_T$ has $n - 1$ d.o.f. as $SS_R = \hat{\beta_1}S_{XY}$ has $1$ d.o.f. ($\hat{\beta_1}$) and $SS_{Res}$ has $n - 2$ d.o.f. as the $n$ samples are constrained by $\hat{\beta_0}$ $\hat{\beta_1}$. We can test $F_0 = \frac{SS_R / 1}{SS_{Res} / (n-2) = \frac{MS_R}{MS_{Res}}} > F_{\alpha, 1, n - 2}$ and as, $E(MS_{Res}) = \sigma^2$ and $E(MS_{Res}) = \sigma^2 + \beta_1^2 S_{XX}$, a large $F_0$ indicates a linear relationship and $F_0$ follows a $F_{1, n - 2}$ distribution with non centrality $\lambda = \frac{\beta_1^2 S_{XX}}{\sigma^2}$. \\
We can convert the t test in 2.3.2 into our F test by squaring it with gives $t_0^2 = \frac{\hat{\beta_1}^2}{MS_{Res}/S_{XX}} = \frac{\hat{\beta_1} S_{XY}}{MS_{Res}} = \frac{MS_R}{MS_{Res}} = F_0$ The square of a t variable with d.o.f. $f$ is a F variable with d.o.f. $1$ and $f$. The t test can test the 1 sided hypothesis but the F test cannot.

\section{2.4 Interval Estimation}
\subsection{2.4.1 Confidence Intervals on $\beta_0$, $\beta_1$ and $\sigma^2$}
Assuming normal and i.i.d. $\epsilon$, we can construct the frequentist confidence interval $\hat{\beta_i} - t_{\alpha/2,n-2} se(\hat{\beta_i}) \le \beta_i \le \hat{\beta_i} + t_{\alpha/2,n-2} se(\hat{\beta_i})$ for $\beta_0$ and $\beta_1$, meaning if we repeatedly sample models, $95\%$ of the CIs will contain the true parameters. For $\sigma^2$, $\frac{(n - 2)MS_{Res}}{\sigma^2}$ is a $\chi^2_{n-2}$ distribution. Thus we can do $P\{\chi^2_{1-\alpha/2,n-2} \le \frac{(n - 2)MS_{Res}}{\sigma^2} \le \chi^2_{\alpha/2,n-2}\} = 1 - \alpha$ and $\frac{(n - 2)MS_{Res}}{\chi^2_{\alpha/2,n-2}} \le \sigma^2 \le \frac{(n - 2)MS_{Res}}{\chi^2_{1-\alpha/2,n-2}}$.

\subsection{2.4.2 Interval Estimation of the Mean Response}
For a regressor $x_0$ within the range of the model, $\mathbb{E}(y|x_0) = \hat{\mu}_{y|x_0} = \hat{\beta_0} + \hat{\beta_1}x_0$ and $Var(\hat{\mu}_{y|x_0}) = \sigma[\frac{1}{n} + \frac{(x_0 - \bar{x})^2}{S_{XX}}]$, so $\hat{\mu}_{y|x_0} - T \le \mathbb{E}(y|x_0) \le \hat{\mu}_{y|x_0} + T$, $T = t_{\alpha/2,n-2}\sqrt{MS_{Res}(\frac{1}{n} + \frac{(x_0 - \bar{x})^2}{S_{XX}})}$. Thus, we should not extrapolate our model as the interval width increases with $|x_0 - \bar{x}|$.

\section{2.5 Prediction of New Observations}
As $\hat{y_0}$ is independent of $y_0$ on., $\psi = y_0 - \hat{y_0}$ is the statistic to base the interval of $y_0$ predicted by $\hat{y_0}$ on. Based on $Var(\psi) = \sigma^2[1 + \frac{1}{n} + \frac{(x_0 - \bar{x})^2}{S_{XX}}]$, our prediction interval is $\hat{y_0} - T \le y_0 \le \hat{y_0} + T$, $T = t_{\alpha/2,n-2}\sqrt{MS_{Res}(1 + \frac{1}{n} + \frac{(x_0 - \bar{x})^2}{S_{XX}})}$. The prediction interval is larger than the CI as we include the noise from the future observation.
\section{2.6 Coefficient of Determination, $R^2$}
$R^2 = \frac{SS_R}{SS_T} = 1 - \frac{SS_{Res}}{SS_T}$ is the proportion of variance explained by the model. It can be inflated by adding more regressors, whilst making the model worse. A polynomial of degree n-1 will perfectly fit a dataset with no repeated $x$. $\mathbb{E}(R^2) = \frac{\beta_1^2S_{XX}/(n - 1)}{\beta_1^2S_{XX}/(n - 1) + \sigma^2}$ shows $R^2$ scales with the spread of $x$, $S_{XX}$.

\section{2.10 Considerations in the Use of Regression}
\textbf{1.} Regression is designed for interpolation not extrapolation. \textbf{2,3.} Extreme $x$ values and $x$ \& $y$ outliers heavily impact the model. \textbf{4.} Correlation does not imply causation. \textbf{5.} If we need to predict $x$, we need to condition our model on the accuracy of $x$'s prediction.

\section{2.11 Regression Through the Origin}
The model $y = \beta_1x + \epsilon$ has least square estimator $\hat{\beta_1} = \frac{\sum y_ix_i}{\sum x_i^2}$ and $\hat{\sigma^2} = MS_{Res} = \frac{\sum y_i^2 - \hat{\beta}\sum y_i x_i}{n - 1}$. The CI on $\beta_1$ is $\hat{\beta_i} \pm t_{\alpha/2,n-1} \sqrt{MS_{Res}/\sum x_i^2}$, the CI on $\mathbb{E}(y|x_0)$ is $\hat{\mu}_{y|x_0} \pm t_{\alpha/2,n-1} \sqrt{x_0^2 MS_{Res} / \sum x_i^2}$ and the prediction interval on $y_0$ is $\hat{y_0} \pm t_{\alpha/2,n-1} \sqrt{MS_{Res} (1 + x_0^2/\sum x_i^2)}$. The intervals on $y$ increase as $x$ increases. The model should only be used when the data is near the origin and we know the $x = 0 \Rightarrow y = 0$.$MS_{Res}$ is a better than $R^2$ at comparing no-intercept and intercept models as $R^2 = \sum \hat{y_i}^2 / \sum y_i^2$ in the no intercept model is the variation around the origin.

\section{2.12 Estimation By Maximum Likelihood}
MLE is the joint distribution of the observations with $ln L(y_i, x_i \beta_0, \beta_1, \sigma^2) \propto - \sum(y_i - \beta_0 - \beta_1 x_i)^2$. $\tilde{\beta_0}$ and $\tilde{\beta_1}$ are identical to the least square estimators whilst $\tilde{\sigma}^2 = \frac{1}{n}\sum(y_i - \tilde{\beta_0} - \tilde{\beta_1}x_i) = \frac{n - 1}{n} \hat{\sigma}^2$ is a biased estimator. MLE is a min variance, unbiased and consistent(close estimate as n increases) estimator and a set of sufficient statistics(contains all information of the model). However, MLE requires full distribution assumption whilst least squares only requires second moment assumptions.

\section{2.13 Case Where the Regressor $x$ is Random}
\subsection{2.13.1 $x$ and $y$ Joint Unknown Distribution}
If $p(y|x) \sim N(\beta_0 + \beta_1x, \sigma^2)$ and $x$ are independent variables not dependent on $\beta_0$, $\beta_1$ and $\sigma^2$, all regression procedures remain unchanged. However, confidence coeff and statistical errors need to be viewed as repeated sampling of $(x_i, y_i)$ and not $y_i$ for fixed $x_i$s. 

\subsection{2.13.2 $x$ and $y$ Joint Normal Distribution}
The conditional distribution is $p(y|x) \sim N(\beta_0 + \beta_1x, \sigma_{y,x})$, $\beta_0 = \mu_y = \mu_x\rho\frac{\sigma_y}{\sigma_x}$ and $\beta_1 = \frac{\sigma_y}{\sigma_x}\rho$, where $\sigma_{y,x} = \sigma_y^2(1 - \rho^2)$ and $\rho = \frac{\sigma_{y,x}}{\sigma_y\sigma_x}$. We see that $\rho = 0 \Rightarrow \beta_1 = 0$, if $y$ and $x$ are independent, there is no linear relationship. We may use MLE to estimate the parameters same as before. \\ 
We can estimate the correlation coefficient $\hat{\rho} = r = S_{xy}/\sqrt{S_{XX}S_T}$ and $\hat{\beta_1} = \sqrt{SS_T/SS_{XX}}r$ shows that the slope is $r$ scaled. $r$ measure the linear association between $y$ and $x$ and is meaningless when $x$ is controllable. $r^2 = R^2$. regression is more powerful than correlation as correlation cannot be used for prediction. \\
We can test $H_0: \rho = 0$ with $t_0 = r\sqrt{n-2}/\sqrt{1-r^2}$ with $|t_0| > t_\alpha/2,n-2$. which is equivalent to testing $H_0: \beta_1 = 0$. The test for $H_0: \rho = \rho_0, \rho_0 \ne 0$ is harder, with $Z = arctanh(r) = 0.5ln[(1+r)/(1-r)]$ being normally distributed with $\mu_Z = arctanh(\rho) = 0.5ln[(1+\rho)/(1-\rho)]$ and $Var_Z=(n-3)^{-1}$ for moderately large samples. So the test is $Z_0 = (arctanh(r)-arctanh(\rho))\sqrt{n-3}$, $|Z_0| > Z_\alpha/2$. The CI of $\rho$ is $tanh(arctanh(r) \pm  Z_\alpha/2/\sqrt{n-3})$ where $tanh(r) = (e^r-e^{-r})/(e^r+e^{-r})$.

\end{multicols*}
\newpage
\begin{multicols*}{3}
\setlength{\premulticols}{1pt}
\setlength{\postmulticols}{1pt}
\setlength{\multicolsep}{1pt}
\setlength{\columnsep}{2pt}

\subsection{3.1 Multiple Regression Models}
A model with $k$ regressors describes a $k$d hyperplane in $k+1$d space(include response). Parameter $\beta_i$ represents the change in response per unit change in $x_i$, holding all other regressor variables constant, a.k.a. partial regression coefficients. Multiple linear regression models are often \textbf{empirical}.

\subsection{3.2 Estimation of The Model Parameters}
\subsubsection{3.2.1 Least-Squares Estimation of Coefficients}
Assuming there are more observations than regressors, the estimator of $\boldsymbol{\beta}$ is $\hat{\boldsymbol{\beta}} = (\X^T\X)^{-1} \mathbf{X}^T\mathbf{y}$, provided the regressors(columns of $\mathbf{X}$) are linearly independent, implying $\mathbf{X}^T\mathbf{X}$ is invertible. $\hat{\mathbf{y}} = \mathbf{X}\hat{\beta} = \mathbf{X}(\X^T\X)^{-1}\mathbf{X}^T\mathbf{y} = \mathbf{H}\mathbf{y}$ shows $\mathbf{H} = \mathbf{X}(\X^T\X)^{-1}\mathbf{X}^T$ maps the observed values to the predicted values. The error term can thus be written as $\mathbf{e} = (\mathbf{I} - \mathbf{H})\mathbf{y}$.

\subsubsection{3.2.2 Geometrical Interpretation of Least-Squares}
We can picture $\mathbf{y}$ as a $n$ dimensional vector and the $p$ $n$ dimensional columns of $\mathbf{X}$ as a $p$ dimensional subspace called the \textbf{estimation space}, so any point in that space can be represented by $\mathbf{X}\boldsymbol{\beta}$. Hence, minimising the error $\boldsymbol{\epsilon}^2 = (\mathbf{y} - \mathbf{X}\boldsymbol{\beta})^2$ is just finding the shortest vector, which will be orthogonal, from vector $\mathbf{y}$ to the estimation space.

\subsubsection{3.2.3 Properties of the Least-Squares Estimators}
The estimators $\E(\hat{\boldbeta}) = \E[(\X^T\X)^{-1}\X^T(\X\boldbeta + \boldeps)] = \boldbeta$ are unbiased and $Cov(\hat{\boldbeta}) = Var[(\X^T\X)^{-1}\X^T\y] = (\X^T\X)^{-1}\X^T Var(y) [(\X^T\X)^{-1}\X^T]^T = \sigma^2(\X^T\X)^{-1}$ is the covariance matrix. $\hat{\boldbeta}$ is a minimum variance estimator of $\boldbeta$. By assuming $\epsilon_i$ are normally distributed, $\hat{\boldbeta}$ is also the MLE of $\boldbeta$ and the MLE is minimum variance too.

\subsubsection{3.2.4 Estimation of $\sigma^2$}
The residual sum of squares is $SS_{Res} = (\mathbf{y} - \mathbf{X}\boldsymbol{\beta})^2 = \y^T\y - \hat{\boldbeta}^T\X^T\y$ with $n-p$ d.o.f. and thus $\hat{\sigma}^2 = MS_{Res} = \frac{SS_{Res}}{n-p}$. Similar to section 2.2.3, $\hat{\sigma}^2$ is model dependent.

\subsubsection{3.2.5 Inadequacy of Scatter Diagrams}
As a scatterplot has only 2 dimensions, we can only visualise the relationship between the response and 1 variable at once. The scatterplot does not fix the other variables, so the hidden regressors can confound the response of the visualised regressor. To properly use scatterplots, we create the entire plot with a fixed value for every other regressors. 

\subsubsection{3.2.6 Maximum-Likelihood Estimation}
Assuming $\boldeps \sim \mathbf{N}(\mathbf{0}, \sigma^2\mathbf{I})$, the log likelihood is $ln L(y_i, x_i \beta_0, \beta_1, \sigma^2) \propto - (\y - \X\hat{\boldbeta})^T(\y - \X\hat{\boldbeta})$ and maximising it is equivalent to solving the least-squares estimator. The MLE for the variance is $\hat{\sigma}^2 = \frac{1}{n}(\y - \X\hat{\boldbeta})^T(\y - \X\hat{\boldbeta})$. These are generalisations from the MLE in section 2.11.

\subsection{3.3 Hypothesis Testing}
\subsubsection{3.3.1 Test for Significance of Regression}
This tests for overall model adequacy, $H_0: \beta_i = 0, \forall i \in [1, k]$ and $H_1: \exists i \in [1, k], \beta_i \neq 0$. To do this, we generalise $SS_T = SS_R + SS_{Res}$ from section 2.3.3 so $SS_R/\sigma^2$ follows a $\chi_k^2$ distribution in $H_0$ and use $F_0 = \frac{SS_R/k}{SS_Res/(n - k - 1)} = \frac{MS_R}{MS_{Res}} > F_{\alpha,k,n-k-1}$ which follows the $F_{k,n-k-1} distribution$. As $\E(MS_{Res} = \sigma^2)$ and $\E(MS_R) = \sigma^2\frac{\lambda}{k}$, where $\beta^*$ are the non intercept parameters, $\X_c$ is regressor mean centered $X$ and $\lambda = \frac{1}{\sigma^2}\boldbeta^{*T}\X_c^T\X_c\boldbeta^*$. If $F_0$ is large, it indicates that we should reject $H_0$ and it has noncentrality parameter $\lambda$. \\
We can also use $R^2$ and $R_{Adj}^2$. As $R^2$ grows when a regressor is added regardless of its contribution, we use $R_{Adj}^2 = 1 - \frac{SS_{Res}/(n-p)}{SS_T/(n-1)}$. As the $SS_T/(n-1)$ is constant, $R_{Adj}^2$. only increases if the addition of a variable reduces $SS_{Res}/(n-p)$.

\subsubsection{3.3.2 Test on Individual Coefficients \& Subsets}
We do a marginal test of individual regressors given the other regressors, $H_0: \beta_i = 0$ \& $H_1: \beta_i \neq 0$, with the test $|t_0| > t_{\alpha/2,n-k-1}$ and statistic $t_0 = \hat{\beta_i} / \sqrt{\hat{\sigma}^2 C_{ii}}$ where $C_{ii}$ is from $(\X^T\X)^{-1}$ \\
For subsets of regressors, we partition $\boldbeta = (\boldbeta_1 | \boldbeta_2)^T$ with $r$ \& $n - r$ elements, then test $H_0: \beta_2 = 0$ \& $H_1: \beta_2 \neq 0$. We fit the reduced model $\y = \X_1\boldbeta_1 + \epsilon_1$ and find its $SS_R(\boldbeta_1) = \hat{\boldbeta_1}\X_1^T\y$ with $p - r$ d.o.f. and the \textbf{extra sum of squares} $SS_R(\boldbeta_2 | \boldbeta_1) = SS_r(\boldbeta) - SS_R(\boldbeta_1)$ with $r$ d.o.f.. We then do the partial F test $F_0 = \frac{SS_R(\boldbeta_2 | \boldbeta_1) / r}{MS_{Res}} > F_{\alpha,r,n-p}$. If $\boldbeta_2 \neq 0$ then $F_0$ has noncentrality parameter $\lambda = \frac{1}{\sigma^2}\boldbeta_2^T\X_2^T[\mathbf{I} - \X_1(\X_1^T\X_1)^{-1}\X_1^T]\X_2\boldbeta_2$. Note if there is multicollinearity in the data, $\lambda$ is near zero even if $\boldbeta_2$ is important as $X_1$ and $X_2$ are near-collinear.

\subsubsection{3.3.3 Special Case of Orthogonal Columns in X}
If the columns of $\X_1$ are orthogonal to those in $\X_2$, we can find $SS_R(\boldbeta_2)$ independent of $\X_1$ as $SS_R(\boldbeta) = SS_R(\boldbeta_1) + SS_R(\boldbeta_2)$.

\subsubsection{3.3.4 Testing the General Linear Hypothesis}
Suppose $H_0: \mathbf{T}\boldbeta = \mathbf{0}$, where $\mathbf{T}$ is an $mxp$ matrix of constants such that $r, r < m$ equations of $\mathbf{T}$ are independent. The full model is  we can test for dependent coefficients and coefficients that are $0$ by obtaining the reduced model $\mathbf{y} = \mathbf{Z}\boldsymbol{\gamma} + \boldeps$, where $\mathbf{Z}$ is a $n x (p-r)$ matrix, by using the $r$ independent equations to solve for the $r$ coefficients of the full model in terms of the remaining $p - r$ coefficients. $SS_{Res}(RM) = \mathbf{y}^T\mathbf{y} - \hat{\boldsymbol{\gamma}}^T\mathbf{Z}^T\hat{\mathbf{y}} \geq SS_{Res}(FM)$ as it has less parameters so a higher d.o.f. for $SS_{Res}$, $n-p+r$ vs $n-p$. We test $H_0: \mathbf{T}\boldbeta=\mathbf{0}$ with $SS_H = SS_{Res}(RM) - SS_{Res}(FM) = \hat{\boldbeta}^T\mathbf{T}^T[\mathbf{T}(\X^T\X)^{-1}\mathbf{T}^T]^{-1}\mathbf{T}\hat{\boldbeta}$ with r d.o.f. and statistic $F_0 = \frac{SS_H/r}{SS_{Res}(FM)/(n-p)} > F_{\alpha,r,n-p}$. We can also test $H_0: \mathbf{T}\boldbeta=\mathbf{c}$ with $SS_H = (\mathbf{T}\hat{\boldbeta}-\mathbf{c})^T[\mathbf{T}(\X^T\X)^{-1}\mathbf{T}^T]^{-1}(\mathbf{T}\hat{\boldbeta}-\mathbf{c})$.

\subsection{3.4 Confidence Intervals in Multiple Regression}
\subsubsection{3.4.1 CIs on the Regression Coefficients}
The marginal distribution of each coefficient is $N(\beta_i, \sigma^2\mathbf{C}_{ii})$ so each statistic is $t_0 = \hat{\beta_i}\beta_i/se(\hat{\beta_i})$ with interval $\hat{\beta_i} \pm t_{\alpha/2,n-p}se(\hat{\beta_i})$ where $se(\hat{\beta_i}) = \sqrt{\sigma^2\mathbf{C}_{ii}}$.

\subsubsection{3.4.2 CIs Estimation of the Mean Response}
For $\mathbf{x}_0$, $\E(y|\mathbf{x}_0) = \mathbf{x}_0^T\boldbeta = \E(\hat{y_0})$ and $Var(\hat{y_0}) = \sigma^2\mathbf{x}_0^T(\X^T\X)^{-1}\mathbf{x}_0$ implies the CI for the mean response $\E(y|\mathbf{x}_0)$ is $\hat{y_0} \pm t_{\alpha/2,n-p}\sqrt{Var(\hat{y_0})}$. The CI's size is good for comparing models, and the further from the mean, the more pronounced the difference.

\subsubsection{3.4.3 Simultaneous CIs on Regression Coefficients}
The CI for all parameters is $(\hat{\boldbeta}\boldbeta)^T\X^T\X(\hat{\boldbeta}\boldbeta)/pMS_{Res} \leq F_{\alpha,p,n-p}$, which creates an $p$ dimensional elliptical region. Alternatively, we can also construct $p$ CIs of $\hat{\boldbeta_i} \pm \Delta se(\hat{\boldbeta_i})$, where $\Delta$ is chosen to ensure the probability of all intervals being correct is the overall CI value. This constructs a bounding rectangle larger than the elliptical before. e.g. we use the Bonferonni method of setting $\Delta = t_{\alpha/2p,n-p}$ for a $1-\alpha$ CI.

\subsection{3.5 Prediction of New Observations}
The CI of $y_0$ from $\mathbf{x}_0$ is $\hat{y_0} \pm t_{\alpha/2,n-p}\sqrt{\hat{\sigma}^2(1 + \mathbf{x}_0^T(\X^T\X)^{-1}\mathbf{x}_0)}$.

\subsection{3.9 Hidden Extrapolation}
In higher dimensions, $\mathbf{x}_0$ can lie in the range of the data but still be an extrapolation as the shape of the data may not be a rectangular. Let the regressor variable hull(RVH) be the smallest convex set containing the data and the largest element on the diagonal of $\mathbf{H}$, $h_{max}$ is on the minimum covering ellipsoid(MCE). Any point $\mathbf{x}_0$, where $x_0^T(\X^T\X)^{-1}x_0 > h_{max}$ is an extrapolation of the model, but \textbf{NOT} vice versa as the ellipsoid is larger than the RVH.

\subsection{3.10 Standardized Regression Coefficients}
The magnitude of coefficients is affected by the data's unit of measurement, hence we need to standardise them to compare coefficients across models. We can do \textbf{unit normal scaling} by $z_{ij} = \frac{x_{ij}-\bar{x}_j}{s_j}$ and $y_i^* = \frac{y_i-\bar{y}}{s_y}$, where $s_j^2 = \frac{1}{n-1}\sum(x_{ij}-\bar{x}_j)^2$
and $s_y^2 = \frac{1}{n-1}\sum(y_i-\bar{y})^2$ are the sample variances. This gives the scaled regressors \& response $\mu = 0$ \& $Var = 1$ and LSE $\hat{\mathbf{b}} = (\mathbf{Z}^T\mathbf{Z})^{-1}\mathbf{Z}^T\mathbf{y}^*$. We can also use \textbf{unit length scaling} with $w_{ij} = \frac{x_{ij}-\bar{x}_j}{\sqrt{s_{jj}}}$ and $y_i^* = \frac{y_i-\bar{y}}{\sqrt{SS_T}}$, where $s_{jj} = \sum(x_{ij}-\bar{x}_j)^2$. Each regressor has $\bar{w}_j = 0$ and length $\sqrt{\sum(w_{ij} - \bar{w}_j)^2}$ and the LSE is $\hat{\mathbf{b}} = (\mathbf{W}^T\mathbf{W})^{-1}\mathbf{W}^T\mathbf{y}^*$. $\mathbf{W}^T\mathbf{W} = \frac{1}{n-1}\mathbf{Z}^T\mathbf{Z}$ is the correlation matrix $\mathbf{r}$ where $r_{ij} = S_{ij}/\sqrt{S_{ii}S_{jj}}$ and $r_{iy} = S_{iy}/\sqrt{S_{ii}S_T}$. Both scaling methods result in the same coefficients, zero the intercept and are related to the original by $\hat{\boldbeta_0} = \hat{b_j}\sqrt{SS_T/SS_{jj}}$.

\subsection{3.11 Multicollinearity}
Near-linear dependence amongst the regression variables is a problem as they reduce the precision of the coefficient LSE. Exact linear dependence results in a singular, non-invertible $\X^T\X$. The diagonals of $\X^T\X$ are the variance inflation factors(VIF) and $VIF_j = 1/(1-R_j^2)$, where $R_j^2$ is obtained by regressing $x_j$ on the other regressors. If $VIF_j = 1$, $x_j$ is orthogonal to all the other regressors. If $VIF_j$ is large, then the regressor is collinear with some other regressors.

\subsection{3.11 Why do Coefficients have the Wrong Sign}
This could happen if: 1. The range of regressors is too small, as $Var(\hat{\boldbeta_i} = \sigma^2/S_ii)$ is inversely proportional to the spread of the regressor. However, spreading the regressor too far could require more complex models for nonlinear responses. 2. Important regressors have been left out, which would have changed the model. 3. Multicollinearity inflates the variance. 4. Rounding or truncating computational errors of an ill conditioned $\X^T\X$ occur.



\end{multicols*}

% \newpage
% \begin{center}
     \Large{\textbf{A quick guide to \LaTeX}} \\
\end{center}
\begin{multicols}{3}
\setlength{\premulticols}{1pt}
\setlength{\postmulticols}{1pt}
\setlength{\multicolsep}{1pt}
\setlength{\columnsep}{2pt}

\section{What is \LaTeX?}
\LaTeX (usually pronounced ``LAY teck,'' sometimes ``LAH teck,'' and never ``LAY tex'') is a mathematics typesetting program that is the standard for most professional mathematics writing. It is based on the typesetting program \TeX\ created by Donald Knuth of Stanford University (his first version appeared in 1978). Leslie Lamport was responsible for creating \LaTeX\, a more user friendly version of \TeX. A team of \LaTeX\ programmers created the current version,  \LaTeX\ 2$\varepsilon$.

\section{Math vs. text vs. functions}
In properly typeset mathematics  variables appear in italics (e.g., $f(x)=x^{2}+2x-3$). The exception to this rule is predefined functions (e.g., $\sin (x)$). Thus it is important to \textbf{always} treat text, variables, and functions correctly. See the difference between $x$ and x, -1 and $-1$, and $sin(x)$ and $\sin(x)$.  

There are two ways to present a mathematical expression--- \emph{inline} or as an \emph{equation}.

\subsection{Inline mathematical expressions}
Inline expressions occur in the middle of a sentence.  To produce an inline expression, place the math expression between dollar signs (\verb!$!).  For example, typing \verb!$90^{\circ}$ is the same as $\frac{\pi}{2}$ radians!  yields $90^{\circ}$ is the same as $\frac{\pi}{2}$ radians.

\subsection{Equations}
Equations are mathematical expressions that are given their own line and are centered on the page.  These are usually used for important equations that deserve to be showcased on their own line or for large equations that cannot fit inline. To produce an inline expression, place the mathematical expression  between the symbols  \verb!\[! and \verb!\]!. Typing \verb!\[x=\frac{-b\pm\sqrt{b^2-4ac}}{2a}\]! yields \[x=\frac{-b\pm\sqrt{b^2-4ac}}{2a}.\]
 
\subsection{Displaystyle} 
To get full-sized inline mathematical expressions  use  \verb!\displaystyle!. Use this sparingly. Typing \verb!I want this $\displaystyle \sum_{n=1}^{\infty}! \verb!\frac{1}{n}$, not this $\sum_{n=1}^{\infty}! \verb!\frac{1}{n}$.! yields\\ I want  this $\displaystyle \sum_{n=1}^{\infty}\frac{1}{n}$, not this $\sum_{n=1}^{\infty}\frac{1}{n}.$


\section{Images}

You can put images (pdf, png, jpg, or gif) in your document. They need to be in the same location as your .tex file when you compile the document. Omit   \verb![width=.5in]! if you want the image to be full-sized.

\verb!\begin{figure}[ht]!\\
\verb!\includegraphics[width=.5in]{imagename.jpg}!\\
\verb!\caption{The (optional) caption goes here.}!\\
\verb!\end{figure}!

\subsection{Text decorations}

Your text can be \textit{italics} (\verb!\textit{italics}!), \textbf{boldface} (\verb!\textbf{boldface}!), or \underline{underlined} (\verb!\underline{underlined}!).

Your math can contain boldface, $\mathbf{R}$ (\verb!\mathbf{R}!), or blackboard bold, $\mathbb{R}$ (\verb!\mathbb{R}!). You may want to used these to express the sets of real numbers ($\mathbb{R}$ or $\mathbf{R}$), integers ($\mathbb{Z}$ or $\mathbf{Z}$), rational numbers ($\mathbb{Q}$ or $\mathbf{Q}$), and natural numbers ($\mathbb{N}$ or $\mathbf{N}$).

To have text appear in a math expression use \verb!\text!. \verb!(0,1]=\{x\in\mathbb{R}:x>0\text{ and }x\le 1\}! yields $(0,1]=\{x\in\mathbb{R}:x>0\text{ and }x\le 1\}$. (Without the \verb!\text! command it treats ``and'' as three variables: $(0,1]=\{x\in\mathbb{R}:x>0 and x\le 1\}$.)



\section{Spaces and new lines}

\LaTeX\ ignores extra spaces and new lines. For example, 

\verb!This   sentence will       look!

\verb!fine after      it is     compiled.!

This   sentence will       look
fine after      it is     compiled.


Leave one full empty line between two paragraphs. Place \verb!\\! at the end of a line to create a new line (but not create a new paragraph).

\verb!This!

\verb!compiles!

~

\verb!like\\!

\verb!this.!

This
compiles 

like\\
this.

Use  \verb!\noindent! to prevent a paragraph from indenting.

\section{Comments}

Use \verb!%! to create a comment. Nothing on the line after the \verb!%! will be typeset. \verb!$f(x)=\sin(x)$ %this is the sine function! yields $f(x)=\sin(x)$%this is the sine function

\section{Delimiters}

\begin{tabular}{lll}
\emph{description} & \emph{command} & \emph{output}\\
parentheses &\verb!(x)! & (x)\\
brackets &\verb![x]! & [x]\\
curly braces& \verb!\{x\}! & \{x\}\\
\end{tabular}

To make your delimiters large enough to fit the content, use them together with \verb!\right! and \verb!\left!. For example, \verb!\left\{\sin\left(\frac{1}{n}\right)\right\}_{n}^! \verb!{\infty}! produces\\ $\displaystyle \left\{\sin\left(\frac{1}{n}\right)\right\}_{n}^{\infty}$.

Curly braces are non-printing characters that are used to gather text that has more than one character. Observe the differences between the four expressions \verb!x^2!, \verb!x^{2}!, \verb!x^2t!, \verb!x^{2t}! when typeset: $x^2$, $x^{2}$, $x^2t$, $x^{2t}$.


\section{Lists}

You can produce ordered and unordered lists.

\begin{tabular}{lll}
\emph{description} & \emph{command} & \emph{output}\\
unordered list&
\begin{tabular}{l}
\verb!\begin{itemize}!\\
\verb!  \item!\\
\verb!  Thing 1!\\
\verb!  \item!\\
\verb!  Thing 2!\\
\verb!\end{itemize}!
\end{tabular}&
\begin{tabular}{l}
$\bullet$ Thing 1\\
$\bullet$ Thing 2
\end{tabular}\\
~\\
ordered list&
\begin{tabular}{l}
\verb!\begin{enumerate}!\\
\verb!  \item!\\
\verb!  Thing 1!\\
\verb!  \item!\\
\verb!  Thing 2!\\
\verb!\end{enumerate}!
\end{tabular}&
\begin{tabular}{l}
1.~Thing 1\\
2.~Thing 2
\end{tabular}
\end{tabular}


\section{Symbols (in \emph{math} mode)}

\subsection{The basics}
\begin{tabular}{lll}
\emph{description} & \emph{command} & \emph{output}\\
addition & \verb!+! & $+$\\
subtraction & \verb!-! & $-$\\
plus or minus & \verb!\pm! & $\pm$\\
multiplication (times) & \verb!\times! & $\times$\\
multiplication (dot) & \verb!\cdot! & $\cdot$\\
division symbol & \verb!\div! & $\div$\\
division (slash) & \verb!/! & $/$\\
circle plus & \verb!\oplus! & $\oplus$\\
circle times & \verb!\otimes! & $\otimes$\\
equal & \verb!=! & $=$\\
not equal & \verb!\ne! & $\ne$\\
less than & \verb!<! & $<$\\
greater than & \verb!>! & $>$\\
less than or equal to & \verb!\le! & $\le$\\
greater than or equal to & \verb!\ge! & $\ge$\\
approximately equal to & \verb!\approx! & $\approx$\\
infinity & \verb!\infty! & $\infty$\\
dots & \verb!1,2,3,\ldots! & $1,2,3,\ldots$\\
dots & \verb!1+2+3+\cdots! & $1+2+3+\cdots$\\
fraction & \verb!\frac{a}{b}! & $\frac{a}{b}$\\
square root & \verb!\sqrt{x}! & $\sqrt{x}$\\
$n$th root & \verb!\sqrt[n]{x}! & $\sqrt[n]{x}$\\
exponentiation & \verb!a^b! & $a^{b}$\\
subscript & \verb!a_b! & $a_{b}$\\
absolute value & \verb!|x|! & $|x|$\\
natural log  & \verb!\ln(x)! & $\ln(x)$\\
logarithms & \verb!\log_{a}b! & $\log_{a}b$\\
exponential function & \verb!e^x=\exp(x)! & $e^{x}=\exp(x)$\\
degree & \verb!\deg(f)! & $\deg(f)$\\
\end{tabular}
\newpage

\subsection{Functions}
\begin{tabular}{lll}
\emph{description} & \emph{command} & \emph{output}\\
maps to & \verb!\to! & $\to$\\
composition& \verb!\circ! & $\circ$\\
piecewise& \verb!|x|=! & \multirow{5}{*}{$\displaystyle |x|=\begin{cases}x&x\ge 0\\-x&x<0\end{cases}$}\\
function&\verb!\begin{cases}!&\\ 
&\verb!x & x\ge 0\\!&\\ 
&\verb!-x & x<0!&\\ 
&\verb!\end{cases}!&
\end{tabular}

\subsection{Greek and Hebrew letters}
\begin{tabular}{llll}
\emph{command} & \emph{output}&\emph{command} & \emph{output}\\
\verb!\alpha! & $\alpha$&\verb!\tau! & $\tau$\\
\verb!\beta! & $\beta$&\verb!\theta! & $\theta$\\
\verb!\chi! & $\chi$&\verb!\upsilon! & $\upsilon$\\
\verb!\delta! & $\delta$&\verb!\xi! & $\xi$\\
\verb!\epsilon! & $\epsilon$&\verb!\zeta! & $\zeta$\\
\verb!\varepsilon! & $\varepsilon$&\verb!\Delta! & $\Delta$\\
\verb!\eta! & $\eta$&\verb!\Gamma! & $\Gamma$\\
\verb!\gamma! & $\gamma$&\verb!\Lambda! & $\Lambda$\\
\verb!\iota! & $\iota$&\verb!\Omega! & $\Omega$\\
\verb!\kappa! & $\kappa$&\verb!\Phi! & $\Phi$\\
\verb!\lambda! & $\lambda$&\verb!\Pi! & $\Pi$\\
\verb!\mu! & $\mu$&\verb!\Psi! & $\Psi$\\
\verb!\nu! & $\nu$&\verb!\Sigma! & $\Sigma$\\
\verb!\omega! & $\omega$&\verb!\Theta! & $\Theta$\\
\verb!\phi! & $\phi$&\verb!\Upsilon! & $\Upsilon$\\
\verb!\varphi! & $\varphi$&\verb!\Xi! & $\Xi$\\
\verb!\pi! & $\pi$&\verb!\aleph! & $\aleph$\\
\verb!\psi! & $\psi$&\verb!\beth! & $\beth$\\
\verb!\rho! & $\rho$&\verb!\daleth! & $\daleth$\\
\verb!\sigma! & $\sigma$&\verb!\gimel! & $\gimel$
\end{tabular}


\subsection{Set theory}
\begin{tabular}{lll}
\emph{description} & \emph{command} & \emph{output}\\
set brackets & \verb!\{1,2,3\}! & $\{1,2,3\}$\\
element of & \verb!\in! & $\in$\\
not an element of & \verb!\not\in! & $\not\in$\\
subset of & \verb!\subset! & $\subset$\\
subset of & \verb!\subseteq! & $\subseteq$\\
not a subset of & \verb!\not\subset! & $\not\subset$\\
contains & \verb!\supset! & $\supset$\\
contains & \verb!\supseteq! & $\supseteq$\\
union & \verb!\cup! & $\cup$\\
intersection & \verb!\cap! & $\cap$\\
big union & 
\verb!\bigcup_{n=1}^{10}A_n! &
$\displaystyle \bigcup_{n=1}^{10}A_{n}$\\
big intersection & \verb!\bigcap_{n=1}^{10}A_n! &$\displaystyle \bigcap_{n=1}^{10}A_{n}$\\
empty set & \verb!\emptyset! & $\emptyset$\\
power set & \verb!\mathcal{P}! & $\mathcal{P}$\\
minimum & \verb!\min! & $\min$\\
maximum & \verb!\max! & $\max$\\
supremum & \verb!\sup! & $\sup$\\
infimum & \verb!\inf! & $\inf$\\
limit superior & \verb!\limsup! & $\limsup$\\
limit inferior & \verb!\liminf! & $\liminf$\\
closure & \verb!\overline{A}! & $\overline{A}$
\end{tabular}

\subsection{Calculus}
\begin{tabular}{lll}
\emph{description} & \emph{command} & \emph{output}\\
derivative & \verb!\frac{df}{dx}! & $\displaystyle \frac{df}{dx}$\\
derivative & \verb!\f'! & $f'$\\
partial derivative & 
\begin{tabular}{l}
\verb!\frac{\partial f}!\\ \verb!{\partial x}! 
\end{tabular}& $\displaystyle \frac{\partial f}{\partial x}$\\
integral & \verb!\int! & $\displaystyle\int$\\
double integral & \verb!\iint! & $\displaystyle\iint$\\
triple integral & \verb!\iiint! & $\displaystyle\iiint$\\
limits & \verb!\lim_{x\to \infty}! & $\displaystyle \lim_{x\to \infty}$\\
summation  & 
\verb!\sum_{n=1}^{\infty}a_n! &
$\displaystyle \sum_{n=1}^{\infty}a_n$\\
product  & 
\verb!\prod_{n=1}^{\infty}a_n! &
$\displaystyle \prod_{n=1}^{\infty}a_n$
\end{tabular}




\subsection{Logic}
\begin{tabular}{lll}
\emph{description} & \emph{command} & \emph{output}\\
not & \verb!\sim! & $\sim$\\
and & \verb!\land! & $\land$\\
or & \verb!\lor! & $\lor$\\
if...then & \verb!\to! & $\to$\\
if and only if & \verb!\leftrightarrow! & $\leftrightarrow$\\
logical equivalence & \verb!\equiv! & $\equiv$\\
therefore & \verb!\therefore! & $\therefore$\\
there exists  & \verb!\exists! & $\exists$\\
for all & \verb!\forall! & $\forall$\\
implies & \verb!\Rightarrow! & $\Rightarrow$\\
equivalent & \verb!\Leftrightarrow! & $\Leftrightarrow$
\end{tabular}

\subsection{Linear algebra}
\begin{tabular}{lll}
\emph{description} & \emph{command} & \emph{output}\\
vector & \verb!\vec{v}! & $\vec{v}$\\
vector & \verb!\mathbf{v}! & $\mathbf{v}$\\
norm & \verb!||\vec{v}||! & $||\vec{v}||$\\
matrix&
\begin{tabular}{l}
\verb!\left[!\\
\verb!\begin{array}{ccc}!\\
\verb!1 & 2 & 3 \\!\\
\verb!4 & 5 & 6\\!\\
\verb!7 & 8 & 0!\\
\verb!\end{array}!\\
\verb!\right]!\end{tabular}&
$\displaystyle \left[\begin{array}{ccc}1 & 2 & 3 \\4 & 5 & 6 \\7 & 8 & 0\end{array}\right]$\\
\\determinant&
\begin{tabular}{l}
\verb!\left|!\\
\verb!\begin{array}{ccc}!\\
\verb!1 & 2 & 3 \\!\\
\verb!4 & 5 & 6 \\!\\
\verb!7 & 8 & 0!\\
\verb!\end{array}!\\
\verb!\right|!
\end{tabular}&
$\displaystyle \left|\begin{array}{ccc}1 & 2 & 3 \\4 & 5 & 6 \\7 & 8 & 0\end{array}\right|$\\
determinant & \verb!\det(A)! & $ \det(A)$\\
trace & \verb!\operatorname{tr}(A)! & $\operatorname{tr}(A)$\\
dimension & \verb!\dim(V)! & $\dim(V)$\\
\end{tabular}

\subsection{Number theory}
\begin{tabular}{lll}
\emph{description} & \emph{command} & \emph{output}\\
divides & \verb!|! & $|$\\
does not divide & \verb!\not |! & $\not |$\\
div & \verb!\operatorname{div}! & $\operatorname{div}$\\
mod & \verb!\mod! & $\operatorname{mod}$\\
greatest common divisor & \verb!\gcd! & $\gcd$\\
ceiling & \verb!\lceil x \rceil! & $\lceil x\rceil$\\
floor & \verb!\lfloor x \rfloor! & $\lfloor x \rfloor$\\
\end{tabular}




\subsection{Geometry and trigonometry}
\begin{tabular}{lll}
\emph{description} & \emph{command} & \emph{output}\\
angle& \verb!\angle ABC! & $\angle ABC$\\
degree& \verb!90^{\circ}! & $90^{\circ}$\\
triangle& \verb!\triangle ABC! & $\triangle ABC$\\
segment& \verb!\overline{AB}! & $\overline{AB}$\\
sine& \verb!\sin! & $\sin$\\
cosine& \verb!\cos! & $\cos$\\
tangent& \verb!\tan! & $\tan$\\
cotangent& \verb!\cot! & $\cot$\\
secant& \verb!\sec! & $\sec$\\
cosecant& \verb!\csc! & $\csc$\\
inverse sine& \verb!\arcsin! & $\arcsin$\\
inverse cosine& \verb!\arccos! & $\arccos$\\
inverse tangent& \verb!\arctan! & $\arctan$\\
\end{tabular}

\section{Symbols (in \emph{text} mode)}

The followign symbols do \textbf{not} have to be surrounded by dollar signs.

\begin{tabular}{lll}
\emph{description} & \emph{command} & \emph{output}\\
dollar sign & \verb!\$! & \$ \\
percent & \verb!\%! & \% \\
ampersand & \verb!\&! & \& \\
pound & \verb!\#! & \# \\
backslash & \verb!\textbackslash! & \textbackslash \\
left quote marks & \verb!``! & `` \\
right quote marks & \verb!''! & '' \\
single left quote  & \verb!`! & ` \\
single right quote  & \verb!'! & ' \\
hyphen & \verb!X-ray! & X-ray\\
en-dash & \verb!pp. 5--15! & pp. 5--15 \\
em-dash & \verb!Yes---or no?! & Yes---or no? 
\end{tabular}

\section{Resources}
Great symbol look-up site: \href{http://detexify.kirelabs.org/}{Detexify}\\
\href{http://amath.colorado.edu/documentation/LaTeX/Symbols.pdf}{\LaTeX\ Mathematical Symbols}\\
\href{ftp://tug.ctan.org/pub/tex-archive/info/symbols/comprehensive/symbols-letter.pdf}{The Comprehensive \LaTeX\ Symbol List}\\ 
\href{http://mirrors.med.harvard.edu/ctan/info/lshort/english/lshort.pdf}{The Not So Short Introduction to \LaTeX\ 2$\varepsilon$}\\
\href{http://www.tug.org/}{TUG: The \TeX\ Users Group}\\
\href{http://www.ctan.org/}{CTAN: The Comprehensive \TeX\ Archive Network}\\
~\\
\LaTeX\ for the Mac: \href{http://www.tug.org/mactex/}{Mac\TeX}\\
\LaTeX\ for the PC: \href{http://www.texniccenter.org/}{{\TeX}nicCenter} and \href{http://miktex.org/}{MiK\TeX}\\
\LaTeX\ online: \href{http://www.writelatex.com/}{WriteLaTeX}.
\vfill
\hrule
~\\
Dave Richeson, Dickinson College, \href{http://divisbyzero.com/}{http://divisbyzero.com/}
\end{multicols}
\end{document}
